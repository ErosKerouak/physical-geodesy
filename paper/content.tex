\section{Por que a força de atração gravitacional $\vec{F}$ é considerada uma quantidade dinâmica?}


A força de atração gravitacional \(\vec{F}\) é considerada uma quantidade dinâmica porque representa uma força específica, isto é, uma força por unidade de massa, ao invés de uma aceleração propriamente dita \citep{sneeuw2006physical}. Enquanto a aceleração é classificada como uma grandeza cinemática, que descreve apenas a mudança na velocidade de um corpo ao longo do tempo, a força específica gravitacional constitui uma grandeza dinâmica, sendo intrinsecamente associada às causas das mudanças de movimento, conforme definido pela Segunda Lei de Newton:

\begin{equation*}
  \label{eq_dinamica}
  \vec{F} = m \vec{a} \text{.}  
\end{equation*}

\noindent
Assim, o vetor força gravitacional específico (\(\vec{a}\)), definido matematicamente por:

\begin{equation*}
  \label{eq_específico}
  \vec{a} = \frac{\vec{F}}{m} \text{,}  
\end{equation*}

\noindent
possui unidades de aceleração (\(m/s_2\)), mas não representa diretamente uma aceleração do corpo, já que a aceleração real sofrida por um corpo depende da combinação de todas as forças atuantes sobre ele, não apenas da gravidade. Em resumo, a natureza dinâmica da força gravitacional decorre do fato de ser uma força específica, determinada pela interação das massas segundo a lei da gravitação universal de Newton:


\begin{equation*}
  \label{eq_gravitação}
  \vec{F} = -\,G \frac{m_1 m_2}{r^3} \vec{r} \text{,}  
\end{equation*}

\noindent
em que \(G\) é a constante gravitacional, \(m_1\) e \(m_2\) são as massas interagentes, e \(\vec{r}\) é o vetor que conecta as duas massas, definindo a direção da força gravitacional \citep{sneeuw2006physical}.
