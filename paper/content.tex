\section{Introdução a Teoria do Potencial}

Entre duas massas, $m$ e $m'$, haverá uma força de atração gravitacional $F$, proporcional ao produto dessas massas e inversamente 
proporcional ao quadrado da distância $l$ que as separa. Considerando que as dimensões das massas sejam pequenas em relação à 
distância entre elas, a força gravitacional pode ser expressa pela seguinte relação:

\begin{equation}
  \label{eq_força_gravitacional}
  F = G\cdot \dfrac{m \cdot m'}{l^2} \hspace{5pt}\text{,}  
\end{equation}

\noindent
em que $G$ representa a constante gravitacional:

\begin{equation*}
  G = 6,674 \times 10^{-11}\ m^3\cdot kg^{-1}\cdot s^{-2}\text{.}  
\end{equation*}

 
Sendo $m$ e $m'$ duas massas pontuais posicionadas, respectivamente, nos pontos $M(\xi,\eta,\zeta)$ e $P(x,y,z)$, define-se o vetor
$\vec{l}$ como vetor posição relativa do ponto $P$ em relação ao ponto $M$, dado por:

\begin{equation}
  \label{eq_vetor_posicao}
  \vec{l} = (x - \xi)\hat{e}_x + (y - \eta)\hat{e}_y + (z - \zeta)\hat{e}_z \text{,}
\end{equation}

\noindent
cujo módulo $l$ é obtido pela expressão:

\begin{equation}
  \label{eq_modulo_v_p}
  l = \sqrt{(x - \xi)^2 + (y - \eta)^2 + (z - \zeta)^2} \text{.}
\end{equation}

\noindent
Os termos $\hat{e}_x$, $\hat{e}_y$ e $\hat{e}_z$ representam os vetores unitários nas direções dos respectivos eixos do sistema 
de coordenadas cartesianas. 

A força de atração gravitacional exercida pela massa $m$ sobre a massa $m'$ é expressa, em forma vetorial, por $\vec{F}$, dada 
pela relação:

\begin{equation}
  \label{eq_força_gravitacional_v}
  \vec{F} = -G \frac{m \cdot m'}{l^2} \hat{e} = -G \frac{m \cdot m'}{l^3}\vec{l}\text{,}
\end{equation}

\noindent
onde $\hat{e} = \frac{\vec{l}}{l}$ representa o vetor unitário orientado segundo a direção de $\vec{l}$. O sinal negativo indica 
que $\vec{F}$ possui direção oposta à de $\vec{l}$, apontando da massa atraída ($m'$) para a massa atrativa ($m$). Devido ao 
caráter recíproco da interação gravitacional, a força exercida pela massa $m'$ sobre a massa $m$ é de mesmo módulo que $\vec{F}$, 
porém de direção oposta.

Para evitar possíveis ambiguidades, podemos definir uma das partículas como atraída, atribuindo-lhe uma massa unitária, 
enquanto a outra será a partícula atrativa. Vamos então considerar que a partícula $m'$ seja a 
partícula atraída, com massa unitária, ou seja, $m' = 1$, enquanto a partícula $m$ será a partícula atrativa, com 
massa $m$. Quando a massa atraida é um ponto material de massa unitaria $m'=1$, (ref{eq_força_gravitacional_v}) torna-se:

\begin{equation}
  \label{eq_f_g_m_unitaria}
  \vec{F} = -G \dfrac{m}{l^3}\vec{l}
\end{equation}

\noindent



\lipsum[1-6]


%%%%%%%%%%%%%%%%%%%%%%%%%%%%%%%%%%%%%%%%%%%%%%%%%%%%%%%%%%%%%%%%%%%%%%%%%%%%%%%
\section{Methodology}

This is Euler's homogeneity equation

\begin{equation}
  \label{eq_euler_homogeneity}
  (x - x_c)\partial_x f + (y - y_c)\partial_y f + (z - z_c)\partial_z f
  = (b - f)\eta
  \ ,
\end{equation}

\noindent
in which $(x_c, y_c, z_c)$ are the coordinates of the magnetic field source,
$b$ is the base level representing a constant shift in the signal, and $\eta$
is the structural index corresponding to the nature of the source. You can
reference the equation in the text like this:
Equation~\ref{eq_euler_homogeneity}.

\lipsum[10-16]



%%%%%%%%%%%%%%%%%%%%%%%%%%%%%%%%%%%%%%%%%%%%%%%%%%%%%%%%%%%%%%%%%%%%%%%%%%%%%%%
\section{Results}


%\begin{figure}[tb]
%\centering
%\includegraphics[width=1\linewidth]{figures/simple-synthetic-data.png}
%\caption{
  %\lipsum[1]
%}
%\label{fig_synthetic_simple_data}
%\end{figure}

\lipsum[5-10]


%%%%%%%%%%%%%%%%%%%%%%%%%%%%%%%%%%%%%%%%%%%%%%%%%%%%%%%%%%%%%%%%%%%%%%%%%%%%%%%
\section{Conclusion}

\lipsum[1-5]


%%%%%%%%%%%%%%%%%%%%%%%%%%%%%%%%%%%%%%%%%%%%%%%%%%%%%%%%%%%%%%%%%%%%%%%%%%%%%%%
\section{Open research}

The Python source code used to produce all results and figures presented here
is available at \url{https://github.com/\GitHubRepository} and
\url{https://doi.org/\ArchiveDOI} under the MIT open-source license.

Here we should cite all of the main software used, like Jupyter, numpy, scipy,
matplotlib, Fatiando, etc.

Cite any data sources as well.



%%%%%%%%%%%%%%%%%%%%%%%%%%%%%%%%%%%%%%%%%%%%%%%%%%%%%%%%%%%%%%%%%%%%%%%%%%%%%%%
\section{Acknowledgements}

We are indebted to the developers and maintainers of the open-source software
without which this work would not have been possible.
Acknowledge any non-author contributors to this study.
Statement about funding.

% Thank the editors and reviewers after review.
