\section{Por que a força de atração gravitacional $\vec{F}$ é considerada uma quantidade dinâmica?}


A força de atração gravitacional \(\vec{F}\) é considerada uma quantidade dinâmica porque representa uma força específica, isto é, uma força por unidade de massa, ao invés de uma aceleração propriamente dita \citep{sneeuw2006physical}. Enquanto a aceleração é classificada como uma grandeza cinemática, que descreve apenas a mudança na velocidade de um corpo ao longo do tempo, a força específica gravitacional constitui uma grandeza dinâmica, sendo intrinsecamente associada às causas das mudanças de movimento, conforme definido pela Segunda Lei de Newton:

\begin{equation*}
  \label{eq_dinamica}
  \vec{F} = m \vec{a} \text{.}  
\end{equation*}

\noindent
Assim, o vetor força gravitacional específico (\(\vec{a}\)), definido matematicamente por:

\begin{equation*}
  \label{eq_específico}
  \vec{a} = \frac{\vec{F}}{m} \text{,}  
\end{equation*}

\noindent
possui unidades de aceleração (\(m/s_2\)), mas não representa diretamente uma aceleração do corpo, já que a aceleração real sofrida por um corpo depende da combinação de todas as forças atuantes sobre ele, não apenas da gravidade. Em resumo, a natureza dinâmica da força gravitacional decorre do fato de ser uma força específica, determinada pela interação das massas segundo a lei da gravitação universal de Newton:


\begin{equation*}
  \label{eq_gravitação}
  \vec{F} = -\,G \frac{m_1 m_2}{r^3} \vec{r} \text{,}  
\end{equation*}

\noindent
em que \(G\) é a constante gravitacional, \(m_1\) e \(m_2\) são as massas interagentes, e \(\vec{r}\) é o vetor que conecta as duas massas, definindo a direção da força gravitacional \citep{sneeuw2006physical}.


\section{Que características ou requisitos deve cumprir o potencial gravitacional $V$ para ser considerado o potencial do vetor força de atração gravitacional $\vec{F}$?}

Para que uma função escalar \( V \) seja considerada o potencial do vetor força de atração gravitacional \( \vec{F} \), ela deve satisfazer uma série de condições físicas e matemáticas fundamentais. Em primeiro lugar, o campo de força \( \vec{F} \) deve ser conservativo, o que implica que ele pode ser representado como o gradiente de um potencial escalar, isto é, \( \vec{F} = \nabla V \). Essa propriedade garante que o trabalho realizado pela força gravitacional entre dois pontos independe do caminho percorrido, sendo uma característica essencial de campos conservativos. Além disso, para que \( V \) seja fisicamente admissível, ele deve ser continuamente diferenciável no espaço considerado, de forma a garantir que suas derivadas parciais — correspondentes às componentes da força — sejam bem definidas e contínuas.

Outra exigência fundamental é que o potencial \( V \) satisfaça a equação de Poisson no interior das massas, dada por \( \Delta V = -4\pi G \rho \), onde \( \rho \) é a densidade de massa, e a equação de Laplace no espaço livre de massa, ou seja, \( \Delta V = 0 \). Essas equações asseguram que o potencial seja compatível com a distribuição de massa que gera o campo gravitacional. Além disso, o potencial deve obedecer à condição de decaimento assintótico no infinito, isto é, deve tender a zero quando a distância ao centro da massa tende ao infinito, como \( V \sim \frac{1}{r} \). Tal comportamento garante a unicidade das soluções dos problemas de contorno no domínio externo, o que é uma exigência matemática relevante para a aplicação de teoremas integrais, como os de Green e Gauss.

O potencial gravitacional também deve ser compatível com o princípio de superposição, segundo o qual o potencial gerado por um sistema de massas é a soma dos potenciais individuais, expressando-se, no caso contínuo, pela integral \( V = G \iiint \frac{\rho(\xi, \eta, \zeta)}{l} \, d\xi d\eta d\zeta \), onde \( l \) é a distância entre o ponto de observação e o elemento de massa. Por fim, é essencial que a derivada espacial do potencial reproduza corretamente a expressão da força gravitacional, de forma que a aplicação do gradiente a \( V \) resulte na forma vetorial da força de Newton: \( \vec{F} = -G \frac{m}{r^3} \vec{r} \). Em síntese, o potencial \( V \) deve ser uma função escalar contínua e diferenciável, cujo gradiente gera o campo gravitacional, cuja Laplaciana está relacionada à densidade de massa por Poisson, que decai adequadamente no infinito e que obedece ao princípio de superposição.