%%%%%%%%%%%%%%%%%%%%%%%%%%%%%%%%%%%%%%%%%%%%%%%%%%%%%%%%%%%%%%%%%%%%%%%%%%%%%%%%%%%%%%%%%%%%%%%%%%%%%%%%%%%%%%%%%%%%%%%%%%%%%%%%%%%%%%%%%%%%%%%%%%%%%%%%%%%%%%%%%%%%%%%%%%%%%%%%%%%%%%%%%%%%%%%%%%%%%%%%%%%%%%%%%%%%%%%%%%%%%%%%%%%


\section{Por que a força de atração gravitacional $\vec{F}$ é considerada uma quantidade dinâmica? Explicar detalhadamente.}


Em mecânica clássica, uma quantidade dinâmica é aquela que atua como causa da variação do estado de movimento de um corpo. De acordo com a Segunda Lei de Newton, o movimento de uma partícula de massa \( m \) sujeita a uma força \( \vec{F} \) é descrito por:

\[
\vec{F} = m \vec{a} \text{,}
\]

\noindent
onde \( \vec{a} \) é a aceleração da partícula. É por isso que a força gravitacional é uma quantidade dinâmica: ela pertence ao domínio das forças, que são as agentes das mudanças de movimento, conforme o princípio fundamental da dinâmica.

Em contraste, aceleração é uma quantidade cinemática, pois descreve apenas a variação da velocidade \(\Delta v\) de um corpo ao longo do tempo \(\Delta t\),

\[
a = \frac{\Delta v}{\Delta t} \text{,}
\]

\noindent
independentemente da causa dessa variação. A aceleração não diz por que o movimento muda, apenas como ele muda.

A chamada força específica (ou aceleração gravitacional), é definida como força por unidade de massa:

\[
\vec{a}_g = \frac{\vec{F}_g}{m} = G \frac{M}{r^2} \hat{r} \text{,}
\]


\noindent
e tem dimensões de aceleração \((\text{m/s}^2)\), mas continua sendo dinâmica em essência, pois representa a intensidade do campo gravitacional gerado por uma massa \(M\), ou seja, a força que esse campo exerce por unidade de massa sobre qualquer corpo colocado em um ponto \(r\).

%%%%%%%%%%%%%%%%%%%%%%%%%%%%%%%%%%%%%%%%%%%%%%%%%%%%%%%%%%%%%%%%%%%%%%%%%%%%%%%%%%%%%%%%%%%%%%%%%%%%%%%%%%%%%%%%%%%%%%%%%%%%%%%%%%%%%%%%%%%%%%%%%%%%%%%%%%%%%%%%%%%%%%%%%%%%%%%%%%%%%%%%%%%%%%%%%%%%%%%%%%%%%%%%%%%%%%%%%%%%%%%%%%%


\section{Que características ou requisitos deve cumprir o potencial gravitacional $V$ para ser considerado o potencial do vetor força de atração gravitacional $\vec{F}$? Demonstrar.}

Para que uma função escalar \( V \) seja considerada o potencial gravitacional associado ao vetor da força de atração gravitacional \( \vec{F} \), ela deve satisfazer um conjunto de propriedades. 

Primeiramente, a função \( V \) deve ser continuamente diferenciável no domínio considerado, ou seja, deve possuir derivadas parciais de primeira ordem contínuas. Essa condição é necessária para que a força gravitacional possa ser definida como o gradiente do potencial, segundo a relação:

\[
\vec{F} = \nabla V = \left( \frac{\partial V}{\partial x}, \frac{\partial V}{\partial y}, \frac{\partial V}{\partial z} \right) \text{.}
\]

\noindent
Além disso, dentro das regiões ocupadas por massa (onde a densidade \( \rho > 0 \)), o potencial \( V \) deve satisfazer a equação de Poisson, dada por:

\[
\Delta V = -4\pi G \rho \text{,}
\]

\noindent
o que garante uma relação direta entre a distribuição de massa e o campo gravitacional resultante.

Por outro lado, fora das massas atrativas, ou seja, nas regiões onde \( \rho = 0 \), o potencial deve satisfazer a equação de Laplace,

\[
\Delta V = 0 \text{,}
\]

\noindent
caracterizando-se como uma função harmônica. 

Também se espera que o potencial gravitacional decaia com a distância. Para um corpo de massa finita, o potencial \( V \) deve tender a zero quando a distância \( l \) ao ponto de observação tende ao infinito, seguindo a forma assintótica:

\[
V \sim \frac{1}{l} \quad \text{quando} \quad l \to \infty \text{,}
\]

\noindent
em conformidade com a física newtoniana, que estabelece que a força gravitacional entre massas decai com o quadrado da distância.

Adicionalmente, a força gravitacional deve ser obtida diretamente como o gradiente do potencial. A aceleração gravitacional (ou força específica) é expressa por:

\[
\vec{g} = \nabla V \text{.}
\]

Por fim, o potencial gravitacional deve permitir a aplicação do princípio da superposição. Isso significa que o potencial total em um ponto qualquer do espaço pode ser calculado como a soma (ou integral) dos potenciais gerados por todas as parcelas de massa do corpo atrator. Essa propriedade resulta da linearidade da equação de Poisson e é expressa pela integral:

\[
V(\vec{r}) = G \iiint \frac{\rho(\vec{r}')}{|\vec{r} - \vec{r}'|} \, dV' \text{.}
\]



%%%%%%%%%%%%%%%%%%%%%%%%%%%%%%%%%%%%%%%%%%%%%%%%%%%%%%%%%%%%%%%%%%%%%%%%%%%%%%%%%%%%%%%%%%%%%%%%%%%%%%%%%%%%%%%%%%%%%%%%%%%%%%%%%%%%%%%%%%%%%%%%%%%%%%%%%%%%%%%%%%%%%%%%%%%%%%%%%%%%%%%%%%%%%%%%%%%%%%%%%%%%%%%%%%%%%%%%%%%%%%%%%%%

\section{Por que fora das massas atrativas o potencial satisfaz a Equação de Laplace e dentro delas a Equação de Poisson? Demonstrar.}

A distinção entre as regiões onde o potencial gravitacional satisfaz a equação de Laplace e aquelas onde ele satisfaz a equação de Poisson fundamenta-se na distribuição da massa no espaço. Considerando uma distribuição contínua de massa, a forma matemática do potencial gravitacional newtoniano \( V(\mathbf{r}) \) em um ponto de observação \( \mathbf{r} \in \mathbb{R}^3 \), gerado por uma densidade de massa \( \rho(\mathbf{r}') \) definida em um volume \( v \subset \mathbb{R}^3 \), é dada por:

\[
V(\mathbf{r}) = G \iiint_{v} \frac{\rho(\mathbf{r}')}{|\mathbf{r} - \mathbf{r}'|} \, d^3\mathbf{r}' \text{,}
\]

\noindent onde \( G \) é a constante da gravitação universal e \( |\mathbf{r} - \mathbf{r}'| \) representa a distância euclidiana entre o ponto de observação \( \mathbf{r} \) e o ponto fonte \( \mathbf{r}' \).

Para investigar as propriedades diferenciais do potencial, aplica-se o operador laplaciano ao termo fundamental \( \frac{1}{|\mathbf{r} - \mathbf{r}'|} \), cuja identidade clássica na teoria do potencial é:

\[
\nabla^2 \left( \frac{1}{|\mathbf{r} - \mathbf{r}'|} \right) = -4\pi \delta(\mathbf{r} - \mathbf{r}') \text{,}
\]

\noindent onde \( \delta \) é a função delta de Dirac. Aplicando o operador laplaciano ao potencial, obtém-se:

\[
\nabla^2 V(\mathbf{r}) = G \iiint_{v} \rho(\mathbf{r}') \nabla^2 \left( \frac{1}{|\mathbf{r} - \mathbf{r}'|} \right) \, d^3\mathbf{r}' = -4\pi G \rho(\mathbf{r}) \text{.}
\]

Esse resultado define a \textbf{equação de Poisson}, válida nos pontos do espaço onde a densidade de massa \( \rho(\mathbf{r}) \) é diferente de zero, ou seja, \textit{no interior das massas atrativas}:

\[
\nabla^2 V(\mathbf{r}) = -4\pi G \rho(\mathbf{r}) \text{.}
\]

No entanto, \textit{fora das massas}, isto é, nos pontos do espaço onde \( \rho(\mathbf{r}) = 0 \), a equação acima reduz-se à chamada \textbf{equação de Laplace}:

\[
\nabla^2 V(\mathbf{r}) = 0 \text{.}
\]


%%%%%%%%%%%%%%%%%%%%%%%%%%%%%%%%%%%%%%%%%%%%%%%%%%%%%%%%%%%%%%%%%%%%%%%%%%%%%%%%%%%%%%%%%%%%%%%%%%%%%%%%%%%%%%%%%%%%%%%%%%%%%%%%%%%%%%%%%%%%%%%%%%%%%%%%%%%%%%%%%%%%%%%%%%%%%%%%%%%%%%%%%%%%%%%%%%%%%%%%%%%%%%%%%%%%%%%%%%%%%%%%%%%

\section{O que representa a Fórmula Integral de Gauss no caso do estudo do potencial gravitacional $V$? Explicar detalhadamente.}


A fórmula integral de Gauss, também conhecida como teorema da divergência, estabelece uma conexão entre a presença de massa em uma região do espaço e o comportamento do campo gravitacional em sua vizinhança. Em termos gerais, essa fórmula afirma que o fluxo total de um campo vetorial \( \vec{F} \) através de uma superfície fechada \( S \), que delimita um volume \( v \), é igual à integral da divergência do campo sobre esse volume:

\[
\iiint_v (\nabla \cdot \vec{F}) \, dv = \iint_S \vec{F} \cdot \hat{n} \, dS \text{,}
\]

\noindent
onde \( \hat{n} \) é o vetor unitário normal à superfície \( S \), orientado para fora. No contexto gravitacional, o campo vetorial \( \vec{F} \) é substituído pela aceleração gravitacional \( \vec{g} \), relacionada ao potencial gravitacional \( V \) por \( \vec{g} = -\nabla V \). Aplicando o teorema da divergência a esse campo, obtemos:

\[
\iiint_v \nabla \cdot \vec{g} \, dv = \iint_S \vec{g} \cdot \hat{n} \, dS \text{.}
\]

\noindent
Utilizando a relação \( \nabla \cdot \vec{g} = -\Delta V \), onde \( \Delta \) é o operador Laplaciano, essa equação pode ser reescrita como:

\[
-\iiint_v \Delta V \, dv = \iint_S \nabla V \cdot \hat{n} \, dS \text{.}
\]

\noindent
De acordo com a equação de Poisson da gravitação, o potencial gravitacional satisfaz \( \Delta V = -4\pi G \rho \), onde \( \rho \) é a densidade de massa. Substituindo essa identidade na equação anterior, obtemos:

\[
\iint_S \nabla V \cdot \hat{n} \, dS = 4\pi G \iiint_v \rho \, dv = 4\pi G M_v \text{,}
\]

\noindent
em que \( M_v \) é a massa total contida no volume \( v \). Essa expressão mostra que o fluxo do campo gravitacional sobre a superfície \( S \) depende exclusivamente da massa interna ao volume, independentemente de sua distribuição ou de quaisquer massas externas. Essa é a essência da lei da gravitação de Gauss, e constitui a versão integral da equação de Poisson.

Fisicamente, isso significa que o campo gravitacional “emerge” da massa, e que a curvatura do potencial gravitacional em uma região está diretamente relacionada à quantidade de massa ali presente. Do ponto de vista matemático, a fórmula de Gauss permite passar da formulação diferencial local (como \( \Delta V = -4\pi G \rho \)) para uma formulação integral global, e vice-versa. 


%%%%%%%%%%%%%%%%%%%%%%%%%%%%%%%%%%%%%%%%%%%%%%%%%%%%%%%%%%%%%%%%%%%%%%%%%%%%%%%%%%%%%%%%%%%%%%%%%%%%%%%%%%%%%%%%%%%%%%%%%%%%%%%%%%%%%%%%%%%%%%%%%%%%%%%%%%%%%%%%%%%%%%%%%%%%%%%%%%%%%%%%%%%%%%%%%%%%%%%%%%%%%%%%%%%%%%%%%%%%%%%%%%%

\section{Por que o potencial gravitacional $1/l$ (i.e. recíproco da distância) é considerado uma função harmônica em toda a região $v$ do espaço $S$, exceto no ponto $P$ onde $l=0$? Demonstrar.}


A função \( \frac{1}{l} \), onde \( l \) representa a distância euclidiana entre dois pontos no espaço, é considerada uma função harmônica em toda a região \( v \subset \mathbb{R}^3 \), exceto no ponto \( P \) onde \( l = 0 \), isto é, onde o ponto de avaliação coincide com a localização da massa pontual. Para demonstrar essa propriedade, parte-se da definição do potencial gravitacional gerado por uma massa pontual \( m \) situada na posição \( (\xi, \eta, \zeta) \):

\[
V(x, y, z) = \frac{Gm}{l} = \frac{Gm}{\sqrt{(x - \xi)^2 + (y - \eta)^2 + (z - \zeta)^2}} \text{,}
\]

\noindent
onde \( l = \sqrt{(x - \xi)^2 + (y - \eta)^2 + (z - \zeta)^2} \). Para simplificação, consideramos a constante \( Gm = 1 \), reduzindo o potencial à forma \( V = \frac{1}{l} \).

A condição para que uma função seja harmônica é que ela satisfaça a equação de Laplace:

\[
\nabla^2 V = \frac{\partial^2 V}{\partial x^2} + \frac{\partial^2 V}{\partial y^2} + \frac{\partial^2 V}{\partial z^2} = 0 \text{.}
\]

\noindent
Aplicando as derivadas parciais a \( V = \frac{1}{l} \), obtêm-se:

\[
\frac{\partial}{\partial x} \left( \frac{1}{l} \right) = -\frac{x - \xi}{l^3} \quad \text{e} \quad \frac{\partial^2}{\partial x^2} \left( \frac{1}{l} \right) = -\frac{1}{l^3} + \frac{3(x - \xi)^2}{l^5} \text{.}
\]

\noindent
Derivações análogas valem para \( y \) e \( z \), resultando na expressão geral:

\[
\nabla^2 \left( \frac{1}{l} \right) =
- \frac{3}{l^3} + \frac{3}{l^5} \left[(x - \xi)^2 + (y - \eta)^2 + (z - \zeta)^2\right] = -\frac{3}{l^3} + \frac{3l^2}{l^5} = 0 \text{.}
\]

\noindent
Portanto, verifica-se que:

\[
\nabla^2 \left( \frac{1}{l} \right) = 0 \quad \text{para } l \neq 0 \text{.}
\]

\noindent
Esta demonstração confirma que \( \frac{1}{l} \) satisfaz a equação de Laplace em todo o domínio exceto no ponto \( P = (\xi, \eta, \zeta) \), onde ocorre uma singularidade (isto é, \( l = 0 \)). Nesse ponto, o potencial torna-se indefinido, divergindo para o infinito. Consequentemente, suas derivadas de segunda ordem também não existem nesse ponto, e a equação de Laplace deixa de ser válida.

%No contexto da teoria das distribuições, é possível mostrar que:

%\[
%\nabla^2 \left( \frac{1}{l} \right) = -4\pi \delta(\mathbf{r} - \mathbf{r}_0) \text{,}
%\]

%\noindent
%em que \( \delta \) é a função delta de Dirac centrada no ponto de singularidade \( \mathbf{r}_0 = (\xi, \eta, \zeta) \). Essa formulação revela que a função \( \frac{1}{l} \) pode ser considerada como uma solução fundamental da equação de Poisson.





%%%%%%%%%%%%%%%%%%%%%%%%%%%%%%%%%%%%%%%%%%%%%%%%%%%%%%%%%%%%%%%%%%%%%%%%%%%%%%%%%%%%%%%%%%%%%%%%%%%%%%%%%%%%%%%%%%%%%%%%%%%%%%%%%%%%%%%%%%%%%%%%%%%%%%%%%%%%%%%%%%%%%%%%%%%%%%%%%%%%%%%%%%%%%%%%%%%%%%%%%%%%%%%%%%%%%%%%%%%%%%%%%%%

\section{O que estabelece o Teorema de Stokes e o Princípio de Dirichlet com relação ao estudo do potencial gravitacional $V$? Explicar detalhadamente.}


O Teorema de Stokes, também conhecido como teorema da unicidade, afirma que uma função harmônica definida em uma região \( R \), limitada por uma superfície fechada \( S \), é unicamente determinada pelos seus valores sobre essa superfície. Mais precisamente, se duas funções \( V_1 \) e \( V_2 \) satisfazem a equação de Laplace,

\[
\nabla^2 V = 0 \quad \text{em } R \text{,}
\]

\noindent
e coincidem ao longo da fronteira \( S \), ou seja,

\[
V_1|_S = V_2|_S \text{,}
\]

\noindent
então \( V_1 = V_2 \) em todo o interior de \( R \). A demonstração baseia-se na primeira identidade de Green, que estabelece a relação

\[
\int_R (V \nabla^2 U + \nabla V \cdot \nabla U) \, dv = \int_S V \frac{\partial U}{\partial n} \, dS \text{,}
\]

\noindent
válida para funções suficientemente regulares. Substituindo \( U = V_1 - V_2 \), que também é harmônica e se anula na fronteira, e tomando \( V = U \), obtém-se

\[
\int_R |\nabla U|^2 \, dv = 0 \text{,}
\]

\noindent
do que se conclui que \( \nabla U = 0 \) em todo \( R \), implicando que \( U \) é constante. Como \( U = 0 \) sobre \( S \), deduz-se que \( U = 0 \) em \( R \), e portanto \( V_1 = V_2 \). Este resultado assegura que o problema de contorno de Dirichlet possui solução única. Em termos físicos, isso significa que o campo de potencial gravitacional externo a uma superfície que delimita uma distribuição de massas é completamente determinado pelos valores do potencial sobre essa superfície.

O Princípio de Dirichlet complementa esse resultado ao afirmar que a solução do problema de contorno realmente existe sob hipóteses adequadas. Dada uma região regular \( R \) e uma função contínua \( f \) definida sobre sua fronteira \( S \), existe uma função harmônica \( V \) tal que

\[
\nabla^2 V = 0 \quad \text{em } R, \quad V|_S = f \text{.}
\]

\noindent
Essa existência pode ser demonstrada por meio de métodos funcionais, como a minimização da energia potencial associada ao campo, ou pela construção explícita de soluções utilizando funções de Green. Na formulação integral clássica, a função de Green permite representar o potencial em função dos dados de contorno. Em particular, a terceira identidade de Green fornece a seguinte expressão para \( V \) em um ponto \( P \) no interior da região:

\[
V(P) = \frac{1}{4\pi} \int_S \left( \frac{1}{l} \frac{\partial V}{\partial n} - V \frac{\partial}{\partial n} \left( \frac{1}{l} \right) \right) \, dS \text{,}
\]

\noindent
onde \( l \) é a distância entre o ponto de integração \( Q \in S \) e o ponto de observação \( P \), e \( \frac{\partial}{\partial n} \) denota a derivada normal externa à superfície. Essa identidade mostra que o valor do potencial no interior da região pode ser inteiramente calculado a partir dos valores de \( V \) e de sua derivada normal ao longo da fronteira.



%%%%%%%%%%%%%%%%%%%%%%%%%%%%%%%%%%%%%%%%%%%%%%%%%%%%%%%%%%%%%%%%%%%%%%%%%%%%%%%%%%%%%%%%%%%%%%%%%%%%%%%%%%%%%%%%%%%%%%%%%%%%%%%%%%%%%%%%%%%%%%%%%%%%%%%%%%%%%%%%%%%%%%%%%%%%%%%%%%%%%%%%%%%%%%%%%%%%%%%%%%%%%%%%%%%%%%%%%%%%%%%%%%%

\section{Qual é a principal diferença entre “harmônico esférico sólido”e “harmônico esférico de superfície”? Explicar detalhadamente.}


A principal diferença entre harmônicos esféricos sólidos e harmônicos esféricos de superfície reside na dimensão do domínio onde são definidos e aplicados. Enquanto os harmônicos de superfície estão restritos a funções definidas sobre uma esfera de raio constante, os harmônicos sólidos estendem-se ao espaço tridimensional, sendo soluções da equação de Laplace em coordenadas esféricas que dependem não apenas dos ângulos, mas também da variável radial. Essa distinção tem implicações diretas sobre a modelagem do potencial gravitacional e sua variação no espaço.

Os harmônicos esféricos de superfície, também chamados de funções de Laplace associadas, são definidos exclusivamente sobre a superfície de uma esfera unitária ou de raio arbitrário constante. Formalmente, são funções da forma

\[
Y_\ell^m(\theta, \lambda) = P_\ell^m(\cos\theta) \, e^{im\lambda} \text{,}
\]

\noindent
onde \( \theta \) é a colatitude, \( \lambda \) é a longitude, \( P_\ell^m \) são os polinômios associados de Legendre, \( \ell \) é o grau e \( m \) é a ordem. Essas funções são ortogonais sobre a esfera e constituem uma base funcional adequada para representar funções escalares bidimensionais, como anomalias gravimétricas medidas ao nível do mar ou da superfície terrestre. 

Por outro lado, os harmônicos esféricos sólidos constituem soluções da equação de Laplace tridimensional em coordenadas esféricas. Essas soluções dependem explicitamente da variável radial \( r \), além dos ângulos \( \theta \) e \( \lambda \), e podem ser expressas na forma geral

\[
V(r, \theta, \lambda) = \sum_{\ell=0}^\infty \sum_{m=0}^\ell \left( A_{\ell m} r^\ell + \frac{B_{\ell m}}{r^{\ell+1}} \right) P_\ell^m(\cos\theta) \cos(m\lambda) + \left( C_{\ell m} r^\ell + \frac{D_{\ell m}}{r^{\ell+1}} \right) P_\ell^m(\cos\theta) \sin(m\lambda) \text{.}
\]

\noindent
Essa classe de funções inclui dois tipos fundamentais: os harmônicos regulares, proporcionais a \( r^\ell \), que são finitos na origem e geralmente usados para representar campos no interior de uma esfera; e os harmônicos singulares, proporcionais a \( r^{-(\ell+1)} \), que são finitos no infinito e usados para representar potenciais no exterior de distribuições de massa. No caso do campo gravitacional da Terra, que é harmônico fora das massas, os termos \( r^{-(\ell+1)} \) são empregados para construir a representação externa do potencial. Assim, o potencial gravitacional externo da Terra é frequentemente expresso como:

\[
V(r, \theta, \lambda) = \frac{GM}{r} \left[ 1 + \sum_{\ell=1}^\infty \sum_{m=0}^\ell \left( \bar{C}_{\ell m} \cos m\lambda + \bar{S}_{\ell m} \sin m\lambda \right) \bar{P}_{\ell m}(\cos\theta) \left( \frac{R}{r} \right)^\ell \right] \text{,}
\]

\noindent
onde \( R \) é o raio de referência (geralmente o raio médio da Terra), e os coeficientes \( \bar{C}_{\ell m} \), \( \bar{S}_{\ell m} \) codificam a distribuição global da massa.

A distinção essencial, portanto, é que os harmônicos de superfície são bidimensionais e restritos a uma esfera, sendo apropriados para modelar fenômenos definidos apenas sobre a superfície, enquanto os harmônicos sólidos são tridimensionais e adequados para representar campos escalares que variam com a distância ao centro, como o potencial gravitacional ou magnético. Na prática, os harmônicos de superfície constituem a parte angular da solução dos harmônicos sólidos, mas a generalização radial é indispensável quando se busca modelar o campo no espaço, fora ou dentro de uma esfera de referência.

%%%%%%%%%%%%%%%%%%%%%%%%%%%%%%%%%%%%%%%%%%%%%%%%%%%%%%%%%%%%%%%%%%%%%%%%%%%%%%%%%%%%%%%%%%%%%%%%%%%%%%%%%%%%%%%%%%%%%%%%%%%%%%%%%%%%%%%%%%%%%%%%%%%%%%%%%%%%%%%%%%%%%%%%%%%%%%%%%%%%%%%%%%%%%%%%%%%%%%%%%%%%%%%%%%%%%%%%%%%%%%%%%%%

\section{Qual é a importância da Integral de Poisson no estudo do potencial gravitacional $V$ e seus diferentes problemas de valor de contorno? Explicar detalhadamente.}

Trata-se de uma solução explícita da equação de Laplace no exterior de uma esfera, quando os valores do potencial são conhecidos na superfície dessa esfera. Considere uma função escalar \( V \) que representa o potencial gravitacional gerado por uma distribuição de massa localizada no interior de uma esfera de raio \( R \). Supondo que \( V \) seja harmônico no exterior da esfera, ou seja, satisfaça a equação de Laplace

\[
\nabla^2 V = 0 \quad \text{para } r > R \text{,}
\]

\noindent
e que seus valores sejam conhecidos sobre a superfície \( r = R \), o problema de contorno de Dirichlet consiste em determinar \( V(r, \theta, \lambda) \) no domínio externo \( r > R \). A Integral de Poisson fornece a solução analítica exata para esse problema:

\[
V(r, \theta, \lambda) = \frac{1}{4\pi} \int_0^{2\pi} \int_0^\pi \frac{R^2 - r^2}{(R^2 + r^2 - 2Rr \cos \psi)^{3/2}} \, f(\theta', \lambda') \sin \theta' \, d\theta' d\lambda' \text{,}
\]

\noindent
onde \( f(\theta', \lambda') \) é o valor conhecido do potencial na superfície esférica de raio \( R \), e \( \psi \) é o ângulo entre os vetores unitários que apontam para os pontos \( (\theta, \lambda) \) e \( (\theta', \lambda') \). A integral é avaliada sobre toda a superfície esférica.

A derivação da fórmula de Poisson tem como ponto de partida a terceira identidade de Green, que expressa a solução do potencial harmônico no domínio externo como uma combinação dos valores do potencial e de suas derivadas normais ao longo da fronteira. Especificamente, se \( V \) é harmônico em um domínio e \( S \) é a fronteira da região, então:

\[
V(P) = \frac{1}{4\pi} \int_S \left( \frac{1}{l} \frac{\partial V}{\partial n} - V \frac{\partial}{\partial n} \left( \frac{1}{l} \right) \right) \, dS \text{,}
\]

\noindent
com \( l \) representando a distância entre o ponto de avaliação \( P \) e o ponto de integração sobre a superfície \( S \), e \( \frac{\partial}{\partial n} \) denotando a derivada normal externa. Quando os valores da derivada normal são desconhecidos (caso típico do problema de Dirichlet), a segunda parcela do integrando é suficiente para determinar a solução, e a integral de Poisson resulta diretamente desse termo.

%%%%%%%%%%%%%%%%%%%%%%%%%%%%%%%%%%%%%%%%%%%%%%%%%%%%%%%%%%%%%%%%%%%%%%%%%%%%%%%%%%%%%%%%%%%%%%%%%%%%%%%%%%%%%%%%%%%%%%%%%%%%%%%%%%%%%%%%%%%%%%%%%%%%%%%%%%%%%%%%%%%%%%%%%%%%%%%%%%%%%%%%%%%%%%%%%%%%%%%%%%%%%%%%%%%%%%%%%%%%%%%%%%%

\section{Por que a Força de Coriolis não é considerada parte do vetor gravidade $\vec{g}$? Explicar detalhadamente.}

A força de Coriolis, embora atue sobre corpos em movimento em referenciais não inerciais como a Terra, não é considerada parte do vetor gravidade \( \vec{g} \) porque sua natureza física e sua forma de atuação são fundamentalmente distintas das componentes que compõem o campo gravitacional.

O vetor gravidade \( \vec{g} \) representa a aceleração total exercida sobre uma massa de prova em repouso com relação à superfície terrestre. Este vetor é composto por duas contribuições principais: a atração gravitacional newtoniana \( \vec{F} \), associada à massa da Terra, e a aceleração centrífuga \( \vec{z} \), resultante da rotação da Terra em torno de seu eixo. Matematicamente, essa definição é expressa por:

\[
\vec{g} = \vec{F} + \vec{z} \text{.}
\]

A atração gravitacional \( \vec{F} \) é uma força de origem física real, associada à interação entre massas, enquanto \( \vec{z} \) é uma força fictícia, mas constante para um dado ponto na superfície da Terra, que atua sobre corpos em repouso devido à rotação do planeta. A força centrífuga pode ser incorporada ao modelo por meio do potencial centrífugo \( \Phi \), permitindo que se defina o potencial normal da gravidade \( U \) como:

\[
U = V + \Phi \text{,}
\]

\noindent
em que \( V \) é o potencial gravitacional. O vetor gravidade então é o gradiente negativo desse potencial total:

\[
\vec{g} = -\nabla U = -\nabla(V + \Phi) \text{.}
\]

Por outro lado, a força de Coriolis não se manifesta sobre corpos em repouso, mas apenas sobre aqueles que se movem em relação ao sistema de referência em rotação. Trata-se de uma força inercial de segunda ordem, proporcional à velocidade do corpo \( \vec{v} \) em relação ao referencial terrestre e ao vetor rotação angular \( \vec{\omega} \) da Terra:

\[
\vec{F}_\text{Coriolis} = -2m (\vec{\omega} \times \vec{v}) \text{.}
\]

\noindent
Essa força é perpendicular tanto ao vetor velocidade quanto ao eixo de rotação da Terra, e sua direção depende do sentido do movimento do corpo. Como ela não atua sobre corpos em repouso, não contribui para o peso aparente medido por instrumentos como gravímetros, que quantificam a força por unidade de massa em repouso com relação à superfície. Além disso, a força de Coriolis não pode ser representada como o gradiente de um potencial escalar, ao contrário da gravidade e da aceleração centrífuga, o que impede sua inclusão formal no potencial total da gravidade.

%%%%%%%%%%%%%%%%%%%%%%%%%%%%%%%%%%%%%%%%%%%%%%%%%%%%%%%%%%%%%%%%%%%%%%%%%%%%%%%%%%%%%%%%%%%%%%%%%%%%%%%%%%%%%%%%%%%%%%%%%%%%%%%%%%%%%%%%%%%%%%%%%%%%%%%%%%%%%%%%%%%%%%%%%%%%%%%%%%%%%%%%%%%%%%%%%%%%%%%%%%%%%%%%%%%%%%%%%%%%%%%%%%%

\section{O potencial de gravidade terrestre $W$ ou também chamado de geopotencial, pode ser considerada uma função harmônica? Explicar detalhadamente.}

O potencial de gravidade terrestre, denotado por \( W \) e comumente chamado de geopotencial, não é, em geral, uma função harmônica, embora contenha uma parcela que o é. A distinção fundamental reside no fato de que \( W \) inclui tanto o potencial gravitacional \( V \), que descreve a atração newtoniana exercida pela massa da Terra, quanto o potencial centrífugo \( \Phi \), que representa a aceleração fictícia devido à rotação do planeta. Assim, o geopotencial é definido por:

\[
W = V + \Phi \text{.}
\]

\noindent
Para compreender por que \( W \) não é, em geral, uma função harmônica, é necessário examinar as propriedades de cada termo.

O potencial gravitacional \( V \) satisfaz a equação de Poisson no interior da Terra (regiões com massa), pois a densidade de massa \( \rho \) não é nula. Sua forma diferencial é:

\[
\nabla^2 V = -4\pi G \rho \text{.}
\]

Em contrapartida, em regiões do espaço onde não há massa, como o exterior da Terra, \( \rho = 0 \), e o potencial gravitacional passa a satisfazer a equação de Laplace:

\[
\nabla^2 V = 0 \quad \text{(em regiões livres de massa).}
\]

\noindent
Logo, o potencial gravitacional é harmônico apenas fora das massas. No interior da Terra, ele não é harmônico, mas sim uma solução da equação de Poisson, o que implica que seu Laplaciano é proporcional à densidade de massa local.

O potencial centrífugo \( \Phi \), por sua vez, tem a forma:

\[
\Phi(\mathbf{r}) = -\frac{1}{2} \omega^2 r_\perp^2 = -\frac{1}{2} \omega^2 (x^2 + y^2) \text{,}
\]

\noindent
onde \( \omega \) é a velocidade angular de rotação da Terra, e \( r_\perp \) é a distância do ponto ao eixo de rotação. Trata-se de uma função não harmônica, uma vez que seu Laplaciano é constante:

\[
\nabla^2 \Phi = -\omega^2 \left( \frac{\partial^2}{\partial x^2} (x^2) + \frac{\partial^2}{\partial y^2} (y^2) \right) = -\omega^2 (2 + 2) = -4\omega^2 \neq 0 \text{.}
\]

Dessa forma, a soma \( W = V + \Phi \) não satisfaz a equação de Laplace nem no interior nem no exterior da Terra. Ainda que \( V \) seja harmônico no exterior, \( \Phi \) não é, e como o Laplaciano é um operador linear, o Laplaciano de \( W \) é:

\[
\nabla^2 W = \nabla^2 V + \nabla^2 \Phi = 0 - 4\omega^2 = -4\omega^2 \neq 0 \text{.}
\]

\noindent
Portanto, o geopotencial \( W \) não é uma função harmônica, nem mesmo fora das massas, devido à contribuição do potencial centrífugo.

%%%%%%%%%%%%%%%%%%%%%%%%%%%%%%%%%%%%%%%%%%%%%%%%%%%%%%%%%%%%%%%%%%%%%%%%%%%%%%%%%%%%%%%%%%%%%%%%%%%%%%%%%%%%%%%%%%%%%%%%%%%%%%%%%%%%%%%%%%%%%%%%%%%%%%%%%%%%%%%%%%%%%%%%%%%%%%%%%%%%%%%%%%%%%%%%%%%%%%%%%%%%%%%%%%%%%%%%%%%%%%%%%%%

\section{Que características deve cumprir uma superfície para ser chamada de “superfície equipotencial” ou “superfície de nível”? Demostrar}

Uma superfície equipotencial, também chamada de superfície de nível, é o conjunto de todos os pontos do espaço onde uma função escalar \( U(\mathbf{r}) \)  (como o potencial gravitacional \( V \) ou o potencial de gravidade \( W \)) assume o mesmo valor constante. Ou seja, a característica essencial que define uma superfície equipotencial é que o potencial não varia ao longo dessa superfície. Matematicamente, isso significa que:

\[
U(\mathbf{r}) = \text{constante} \quad \text{para todo } \mathbf{r} \in S \text{,}
\]

\noindent
onde \( S \) é a superfície em questão.

Para que uma superfície seja considerada equipotencial, ela deve cumprir as seguintes condições matemáticas e físicas:

\begin{enumerate}
   \item \textbf{O gradiente do potencial é perpendicular à superfície em todos os seus pontos.} \\
   Essa propriedade decorre diretamente do fato de que o vetor gradiente aponta na direção de maior variação da função escalar. Como não há variação do potencial ao longo da superfície equipotencial, o vetor gradiente de \( U \) deve ser normal à superfície em todos os pontos:
   \[
   \nabla U \perp T_p(S) \quad \text{para todo } p \in S \text{,}
   \]
   onde \( T_p(S) \) representa o plano tangente à superfície \( S \) no ponto \( p \).

   \item \textbf{A força associada ao campo é perpendicular à superfície.} \\
   O campo de força correspondente é obtido como o gradiente negativo do potencial:
   \[
   \vec{F} = -\nabla U \text{.}
   \]
   Isso implica que a força atua na direção normal à superfície equipotencial, ou seja, não existe componente tangencial da força sobre essa superfície. Em termos físicos, isso significa que uma massa colocada sobre uma superfície equipotencial não se moveria espontaneamente ao longo dela (caso não haja outras forças atuando), pois não haveria força resultante nessa direção.

   \item \textbf{A diferencial do potencial é nula para deslocamentos infinitesimais ao longo da superfície.} \\
   Seja \( d\mathbf{r} \) um deslocamento infinitesimal tangente à superfície, então:
   \[
   dU = \nabla U \cdot d\mathbf{r} = 0 \text{.}
   \]
   Isso confirma que o valor do potencial permanece constante ao longo de qualquer trajetória contida na superfície.
\end{enumerate}


Essas propriedades podem ser sintetizadas da seguinte forma: uma superfície \( S \) é equipotencial se e somente se o campo vetorial derivado de \( U \) (seja ele o campo gravitacional \( \vec{g} = -\nabla W \), ou o campo de atração \( \vec{F} = -\nabla V \)) for sempre ortogonal à superfície e o potencial não variar sobre ela.

%%%%%%%%%%%%%%%%%%%%%%%%%%%%%%%%%%%%%%%%%%%%%%%%%%%%%%%%%%%%%%%%%%%%%%%%%%%%%%%%%%%%%%%%%%%%%%%%%%%%%%%%%%%%%%%%%%%%%%%%%%%%%%%%%%%%%%%%%%%%%%%%%%%%%%%%%%%%%%%%%%%%%%%%%%%%%%%%%%%%%%%%%%%%%%%%%%%%%%%%%%%%%%%%%%%%%%%%%%%%%%%%%%%

\section{Que relação é possível estabelecer através da Equação Generalizada de Bruns? Explicar detalhadamente.}

A Equação Generalizada de Bruns estabelece uma relação fundamental entre a altura geoidal \( N \) e a o potencial anomalo \( T \), que representa a diferença entre o potencial de gravidade real da Terra \( W \) e o potencial de um modelo matemático idealizado \( U \), conhecido como potencial normal. A equação é expressa da seguinte forma:

\[
N = \frac{T}{\gamma} = \frac{W - U}{\gamma} \text{,}
\]

\noindent
onde: \( N \) é a altura geoidal; \( T = W - U \) é o potencial anomalo; e \( \gamma \) é o módulo da gravidade normal no ponto considerado.


A equação mostra que a separação entre o geóide e o elipsoide (representada por \( N \)) depende da intensidade da anomalia do potencial no ponto considerado e da gravidade normal. Ou seja, quanto maior a diferença entre o campo gravitacional real e o idealizado, maior será o desvio da superfície física da Terra em relação ao modelo matemático.

Essa relação decorre de uma linearização da equação do potencial ao longo da direção da vertical física, assumindo que \( W = U \) no geóide (por definição, o geóide é uma superfície equipotencial do campo real), e que a variação de \( U \) ao longo da vertical próxima ao elipsoide pode ser aproximada por uma derivada constante. A partir disso, pode-se escrever:

\[
W(P) = U(Q) \approx U(P) + \frac{\partial U}{\partial h} N = U(P) + \gamma N \text{,}
\]

\noindent
de onde se isola:

\[
N = \frac{W(P) - U(P)}{\gamma} = \frac{T(P)}{\gamma} \text{.}
\]

%A equação generalizada de Bruns tem diversas aplicações práticas:

%\begin{enumerate}
%   \item \textbf{Determinação do geóide a partir de dados gravimétricos:} \\
%   Como \( T \) pode ser estimado por meio de medições gravimétricas e modelos do potencial normal, a %quação permite calcular \( N \) e, assim, mapear a forma do geóide.

%   \item \textbf{Transformação entre alturas:} \\
%   A equação é usada para converter alturas geométricas (obtidas por GNSS em relação ao elipsoide) em alturas físicas (ortométricas ou dinâmicas), que se referem ao nível médio do mar, por meio da relação:
%   \[
%   H = h - N \text{,}
%   \]
%   onde \( h \) é a altura elipsoidal e \( H \) é a altura ortométrica.

%   \item \textbf{Avaliação de modelos geopotenciais:} \\
%   A anômala do potencial \( T \) é sensível à qualidade do modelo normal adotado. Assim, a equação de Bruns também serve para avaliar a fidelidade dos modelos de gravidade utilizados.
%\end{enumerate}

%%%%%%%%%%%%%%%%%%%%%%%%%%%%%%%%%%%%%%%%%%%%%%%%%%%%%%%%%%%%%%%%%%%%%%%%%%%%%%%%%%%%%%%%%%%%%%%%%%%%%%%%%%%%%%%%%%%%%%%%%%%%%%%%%%%%%%%%%%%%%%%%%%%%%%%%%%%%%%%%%%%%%%%%%%%%%%%%%%%%%%%%%%%%%%%%%%%%%%%%%%%%%%%%%%%%%%%%%%%%%%%%%%%

\section{Explicar detalhadamente as relações físicas existentes entre os harmônicos de baixo grau e o cálculo do centro de massas e os momentos e produtos de inercia da Terra.}

As componentes de baixo grau da expansão em harmônicos esféricos do potencial gravitacional da Terra contêm informações físicas sobre a distribuição global de massa do planeta. Em particular, os coeficientes harmônicos dos graus \( \ell = 1 \) e \( \ell = 2 \) estão diretamente relacionados ao centro de massas da Terra e aos seus momentos e produtos de inércia principais. 

Os termos de grau \( \ell = 1 \) da expansão harmônica esférica representam translações do sistema de coordenadas em relação ao centro de massa da Terra. Na forma geral do potencial gravitacional externo \( V \), expresso por harmônicos esféricos (para \( r > R \), com \( R \) sendo o raio de referência):

\[
V(r, \theta, \lambda) = \frac{GM}{r} \left[ 1 + \sum_{\ell=1}^{\infty} \left( \frac{R}{r} \right)^\ell \sum_{m=0}^{\ell} \left( \bar{C}_{\ell m} \cos m\lambda + \bar{S}_{\ell m} \sin m\lambda \right) \bar{P}_{\ell m}(\cos \theta) \right] \text{,}
\]

\noindent
os coeficientes \( \bar{C}_{10} \), \( \bar{C}_{11} \), e \( \bar{S}_{11} \) correspondem ao grau 1. Fisicamente, esses coeficientes são proporcionais às componentes da posição do centro de massa da Terra com relação ao sistema de coordenadas adotado. Caso o sistema de coordenadas esteja corretamente centrado no centro de massa da Terra, então os coeficientes de grau 1 devem ser nulos:

\[
\bar{C}_{10} = \bar{C}_{11} = \bar{S}_{11} = 0 \text{.}
\]

\noindent
Portanto, a presença de termos de grau 1 indica que o sistema de referência adotado (por exemplo, o centro do elipsoide de referência) não coincide com o centro de massa físico da Terra. Essa é uma verificação importante para validar o alinhamento dos sistemas de referência em modelos geopotenciais globais.

Os coeficientes harmônicos de grau \( \ell = 2 \) (que incluem \( \bar{C}_{20} \), \( \bar{C}_{21} \), \( \bar{S}_{21} \), \( \bar{C}_{22} \) e \( \bar{S}_{22} \)) estão diretamente relacionados à matriz de inércia da Terra, que descreve como a massa está distribuída com relação ao centro de massa e aos eixos principais.

A matriz de inércia \( \mathbf{I} \) da Terra é definida como:

\[
\mathbf{I} = 
\begin{bmatrix}
I_{xx} & -I_{xy} & -I_{xz} \\
-I_{yx} & I_{yy} & -I_{yz} \\
-I_{zx} & -I_{zy} & I_{zz}
\end{bmatrix}
\text{,}
\]

\noindent
onde os termos diagonais \( I_{xx}, I_{yy}, I_{zz} \) são os momentos principais de inércia, e os termos fora da diagonal \( I_{xy}, I_{xz}, I_{yz} \) são os produtos de inércia.

Esses componentes de inércia estão relacionados com os coeficientes do potencial gravitacional da seguinte forma (valores simplificados e normalizados para o caso do grau 2):

\begin{itemize}
   \item \textbf{O coeficiente \( \bar{C}_{20} \):} \\
   Está associado à diferença entre os momentos polares e equatoriais de inércia. Ele é responsável por descrever o achatamento polar da Terra:
   \[
   \bar{C}_{20} \propto \frac{1}{2} \left( 2I_{zz} - I_{xx} - I_{yy} \right)
   \]
   Quando positivo (na convenção antiga), indica que a Terra é mais achatada nos polos do que no equador, como ocorre de fato.

   \item \textbf{Os coeficientes \( \bar{C}_{21} \) e \( \bar{S}_{21} \):} \\
   Descrevem a inclinação do eixo de rotação em relação aos eixos principais de inércia (ou à orientação do sistema de referência). Eles estão ligados aos produtos de inércia \( I_{xz} \) e \( I_{yz} \):
   \[
   \bar{C}_{21} \propto I_{xz}, \quad \bar{S}_{21} \propto I_{yz}
   \]

   \item \textbf{Os coeficientes \( \bar{C}_{22} \) e \( \bar{S}_{22} \):} \\
   Estão ligados à assimetria equatorial da Terra, ou seja, às diferenças de distribuição de massa no plano equatorial. São relacionados a:
   \[
   \bar{C}_{22} \propto \frac{1}{4}(I_{xx} - I_{yy}), \quad \bar{S}_{22} \propto I_{xy}
   \]
\end{itemize}




%A análise desses termos permite estimar a orientação e os eixos principais da massa terrestre, bem como estudar o comportamento dinâmico do planeta — como precessão, nutação e variações no eixo de rotação (movimento de Chandler, por exemplo).

%O estudo dos harmônicos de baixo grau revela informações fundamentais sobre a estrutura da Terra como corpo físico:

%- Permite identificar o centro de massa e verificar o alinhamento dos sistemas de referência;
%- Permite calcular os momentos de inércia, essenciais para a compreensão da dinâmica de rotação da Terra;
%- Ajuda a inferir a distribuição global de massas, essencial em estudos de estrutura interna da Terra, oceanografia, glaciologia e modelagem de marés;
%- É base para a determinação precisa do campo gravitacional terrestre global, a partir de missões como GRACE e GOCE, que medem diretamente essas variações.

%%%%%%%%%%%%%%%%%%%%%%%%%%%%%%%%%%%%%%%%%%%%%%%%%%%%%%%%%%%%%%%%%%%%%%%%%%%%%%%%%%%%%%%%%%%%%%%%%%%%%%%%%%%%%%%%%%%%%%%%%%%%%%%%%%%%%%%%%%%%%%%%%%%%%%%%%%%%%%%%%%%%%%%%%%%%%%%%%%%%%%%%%%%%%%%%%%%%%%%%%%%%%%%%%%%%%%%%%%%%%%%%%%%

\section{O que quer dizer que o potencial gravitacional $V$ é simétrico em relação à rotação terrestre quando falamos de esferopotencial? Explicar detalhadamente.}

Quando se afirma que o potencial gravitacional \( V \) é simétrico em relação à rotação terrestre no contexto do esferopotencial, quer-se dizer que esse potencial é invariável sob rotações em torno do eixo da Terra, ou seja, não depende da longitude geográfica \( \lambda \). Em termos matemáticos, essa propriedade de simetria axial se expressa pela condição:

\[  
\frac{\partial V}{\partial \lambda} = 0 \text{,}
\]

\noindent
o que implica que o valor do potencial em um ponto qualquer depende exclusivamente da distância radial ao centro da Terra \( r \) e da colatitude \( \theta \) (ou latitude), mas não varia ao longo dos meridianos. Essa invariância longitudinal está diretamente relacionada ao modelo idealizado adotado no esferopotencial, no qual a Terra é representada como um corpo cuja massa está distribuída de maneira perfeitamente simétrica em torno de seu eixo de rotação.

Tal simetria tem uma consequência direta na forma da equação do potencial gravitacional quando este é expandido em harmônicos esféricos. A expansão geral do potencial gravitacional externo da Terra, válido para pontos fora da massa terrestre, é dada por:

\[
V(r, \theta, \lambda) = \frac{GM}{r} \left[1 + \sum_{\ell=1}^{\infty} \left( \frac{R}{r} \right)^\ell \sum_{m=0}^\ell \left( \bar{C}_{\ell m} \cos m\lambda + \bar{S}_{\ell m} \sin m\lambda \right) \bar{P}_{\ell m}(\cos \theta) \right] \text{,}
\]

\noindent
onde \( GM \) é o produto da constante gravitacional pela massa da Terra, \( R \) é o raio de referência, \( \bar{P}_{\ell m} \) são os polinômios de Legendre normalizados, e os coeficientes \( \bar{C}_{\ell m} \), \( \bar{S}_{\ell m} \) representam a contribuição da distribuição de massa para cada grau \( \ell \) e ordem \( m \). No modelo esferopotencial, os termos de ordem \( m \geq 1 \) desaparecem devido à ausência de variações longitudinais, restando apenas os termos de ordem zero (os chamados termos zonais), de modo que a expressão se reduz para:

\[
V(r, \theta) = \frac{GM}{r} \left[1 - \sum_{\ell=1}^{\infty} J_\ell \left( \frac{R}{r} \right)^\ell P_\ell(\cos \theta) \right] \text{,}
\]

\noindent
em que \( J_\ell = -\bar{C}_{\ell 0} \) são os coeficientes zonais e \( P_\ell \) são os polinômios de Legendre de grau \( \ell \). O termo mais significativo nessa expansão é o de grau \( \ell = 2 \), isto é, \( J_2 \), que representa o achatamento da Terra nos polos devido à rotação. A presença desse termo reflete a forma oblata do planeta, caracterizada por um raio equatorial maior do que o raio polar.

%%%%%%%%%%%%%%%%%%%%%%%%%%%%%%%%%%%%%%%%%%%%%%%%%%%%%%%%%%%%%%%%%%%%%%%%%%%%%%%%%%%%%%%%%%%%%%%%%%%%%%%%%%%%%%%%%%%%%%%%%%%%%%%%%%%%%%%%%%%%%%%%%%%%%%%%%%%%%%%%%%%%%%%%%%%%%%%%%%%%%%%%%%%%%%%%%%%%%%%%%%%%%%%%%%%%%%%%%%%%%%%%%%%

\section{Quais são as chamadas de “constantes de Stokes” e qual é sua principal funcionalidade no cálculo do esferopotencial? Explicar detalhadamente.}

As chamadas "constantes de Stokes" são os coeficientes que aparecem na expansão do potencial gravitacional externo da Terra em termos de funções harmônicas zonais, sob a hipótese de simetria axial do campo de gravidade, ou seja, na formulação do esferopotencial. Essas constantes representam a contribuição de diferentes ordens de achatamento e de variações simétricas da massa da Terra em torno do seu eixo de rotação, sendo utilizadas para modelar a forma do potencial gravitacional idealizado, chamado de potencial normal \( U \).

No contexto do esferopotencial, assume-se que o potencial gravitacional da Terra, no exterior da massa (isto é, para \( r > R \), onde \( R \) é o raio de referência do modelo), satisfaz a equação de Laplace e apresenta simetria axial em torno do eixo de rotação. Com isso, a expansão do potencial gravitacional se reduz a uma série contendo apenas os termos de ordem \( m = 0 \), ou seja, os termos zonais da expansão em harmônicos esféricos. A forma geral do potencial gravitacional externo se expressa como:

\[
V(r, \theta) = \frac{GM}{r} \left[ 1 - \sum_{\ell=1}^\infty J_\ell \left( \frac{R}{r} \right)^\ell P_\ell(\cos \theta) \right] \text{,}
\]

\noindent
onde \( GM \) representa o produto da constante gravitacional universal \( G \) pela massa da Terra \( M \); \( r \) é a distância radial ao centro da Terra; \( \theta \) é a colatitude, isto é, o ângulo medido a partir do polo norte; \( P_\ell(\cos \theta) \) são os polinômios de Legendre de grau \( \ell \); e \( J_\ell \) são os coeficientes zonais de grau \( \ell \), tradicionalmente denominados constantes de Stokes.

Essas constantes, \( J_2, J_4, J_6, \ldots \), quantificam o desvio do campo gravitacional da Terra em relação ao campo de um corpo esférico perfeito. A constante mais significativa é \( J_2 \), que representa o achatamento equatorial da Terra devido à rotação. Ela está diretamente relacionada à diferença entre os raios equatorial e polar do elipsoide terrestre, e é responsável pela maior parte da discrepância entre a gravidade medida na superfície e aquela que seria esperada em uma esfera homogênea. Os termos de grau mais alto, como \( J_4 \), \( J_6 \), e assim por diante, descrevem variações mais sutis da forma e da estrutura interna da Terra, incluindo irregularidades como a diferença de densidade entre o manto e o núcleo.

%%%%%%%%%%%%%%%%%%%%%%%%%%%%%%%%%%%%%%%%%%%%%%%%%%%%%%%%%%%%%%%%%%%%%%%%%%%%%%%%%%%%%%%%%%%%%%%%%%%%%%%%%%%%%%%%%%%%%%%%%%%%%%%%%%%%%%%%%%%%%%%%%%%%%%%%%%%%%%%%%%%%%%%%%%%%%%%%%%%%%%%%%%%%%%%%%%%%%%%%%%%%%%%%%%%%%%%%%%%%%%%%%%%

\section{Explique detalhadamente as principais características e/ou diferenças entre geopotencial, esferopotencial e potencial anômalo.}

O geopotencial, geralmente denotado por \( W \), representa o potencial total de gravidade real da Terra em um ponto do espaço. Ele é definido como a soma do potencial gravitacional \( V \), que descreve a atração newtoniana causada pela massa da Terra, e do potencial centrífugo \( \Phi \), que resulta da rotação do planeta. Sua expressão é:

\[
W = V + \Phi \text{.}
\]

\noindent
O geopotencial é a função escalar cujo gradiente negativo define o vetor gravidade efetivo \( \vec{g} = -\nabla W \), ou seja, ele determina tanto a direção quanto a intensidade da gravidade medida em repouso na superfície da Terra. Como \( \Phi \) depende da posição em relação ao eixo de rotação e não é uma função harmônica, o geopotencial tampouco é uma função harmônica. Além disso, as superfícies de nível constante de \( W \), chamadas de superfícies equipotenciais, incluem o geóide, que é definido como a superfície equipotencial que mais se aproxima do nível médio dos mares. O geopotencial é, portanto, uma função realista, mas complexa, que incorpora todas as irregularidades e assimetrias da distribuição de massa do planeta.

O esferopotencial, por outro lado, é uma representação simplificada e idealizada do campo gravitacional da Terra. Também chamado de potencial normal, geralmente simbolizado por \( U \), ele corresponde ao potencial gravitacional de uma Terra fictícia, simétrica em relação ao seu eixo de rotação, e com massa distribuída de forma regular em um elipsoide de revolução. Sua principal característica é a simetria axial, o que implica que ele não depende da longitude \( \lambda \), apenas da distância radial \( r \) e da colatitude \( \theta \). Essa simetria permite que \( U \) seja expresso como uma expansão em harmônicos esféricos contendo apenas os termos zonais (com \( m = 0 \)):

\[
U(r, \theta) = \frac{GM}{r} \left[1 - \sum_{\ell=1}^\infty J_\ell \left( \frac{R}{r} \right)^\ell P_\ell(\cos \theta) \right] \text{,}
\]

\noindent
onde os coeficientes \( J_\ell \), chamados de constantes de Stokes, representam a contribuição de diferentes ordens de achatamento e assimetria axial. O esferopotencial é, por construção, uma função analítica e harmônica no exterior da Terra, o que o torna matematicamente conveniente como modelo de referência para comparar com o campo real.

A diferença entre o campo real e esse campo idealizado é medida por uma terceira função: o potencial anômalo, denotado por \( T \). Essa função é definida como a diferença entre o geopotencial real \( W \) e o esferopotencial \( U \):

\[
T = W - U \text{.}
\]

\noindent
O potencial anômalo expressa, portanto, todas as irregularidades do campo de gravidade real em relação ao modelo idealizado. Ele reflete variações laterais da densidade terrestre, a influência de montanhas, fossas oceânicas, bacias sedimentares, bem como quaisquer outros desvios em relação à simetria axial assumida no modelo normal. O potencial anômalo é o campo fundamental nas análises gravimétricas, pois está diretamente ligado à anômala da gravidade, à deflexão da vertical e à altura geoidal, sendo usado para estudar a estrutura interna da Terra e para refinar modelos locais ou regionais do geóide.

%%%%%%%%%%%%%%%%%%%%%%%%%%%%%%%%%%%%%%%%%%%%%%%%%%%%%%%%%%%%%%%%%%%%%%%%%%%%%%%%%%%%%%%%%%%%%%%%%%%%%%%%%%%%%%%%%%%%%%%%%%%%%%%%%%%%%%%%%%%%%%%%%%%%%%%%%%%%%%%%%%%%%%%%%%%%%%%%%%%%%%%%%%%%%%%%%%%%%%%%%%%%%%%%%%%%%%%%%%%%%%%%%%%

\section{O que estabelece o Teorema de Clairaut? Explicar detalhadamente.}

O Teorema de Clairaut estabelece uma relação entre a variação da gravidade com a latitude e a forma da Terra, assumida como um elipsoide de revolução em equilíbrio hidrostático, vinculando esse comportamento ao seu achatamento e à rotação do planeta.

A Terra, sendo um corpo em rotação, não é uma esfera perfeita, mas sim ligeiramente oblata: seu raio equatorial é maior que o polar. Essa forma oblata, associada à força centrífuga causada pela rotação, faz com que a gravidade varie com a latitude, sendo menor no equador e maior nos polos. O Teorema de Clairaut quantifica essa variação, estabelecendo que, sob certas hipóteses razoáveis (massa distribuída de forma simétrica em torno do eixo de rotação, equilíbrio hidrostático e pequena excentricidade), a gravidade \( g \) em um ponto sobre a superfície da Terra é dada por:

\[
g(\phi) = g_e \left( 1 + \beta \sin^2 \phi \right) \text{,}
\]

\noindent
onde \( g(\phi) \) é a gravidade normal em uma latitude geodésica \( \phi \); \( g_e \) é a gravidade no equador; \( \beta \) é um fator que depende do achatamento \( f \) da Terra e da velocidade angular \( \omega \); e \( \sin^2 \phi \) expressa a dependência da gravidade com a latitude.

A forma mais conhecida e generalizada da expressão é:

\[
g(\phi) = g_e \left[ 1 + \left( \frac{5}{2}m - f \right)\sin^2 \phi \right]
\]

\noindent
em que \( f = \frac{a - b}{a} \) é o achatamento do elipsoide, sendo \( a \) e \( b \) os semieixos maior (equatorial) e menor (polar), respectivamente, e \( m = \frac{\omega^2 a^2 b}{GM} \) é o parâmetro dinâmico da Terra, que expressa a razão entre a força centrífuga e a força gravitacional.

%A importância do Teorema de Clairaut é que ele permite, com base em medidas da gravidade em diferentes latitudes, estimar o achatamento da Terra e verificar a consistência entre a forma física da Terra e o modelo teórico de equilíbrio hidrostático. Na prática, ele também fornece uma aproximação de primeira ordem para o cálculo da gravidade normal sobre o elipsoide, o que é essencial para correções gravimétricas, determinação de alturas físicas e para a formulação do potencial normal \( U \).

%Esse teorema é derivado a partir das condições de equilíbrio de um fluido em rotação e do fato de que, no equilíbrio hidrostático, a superfície física (como o nível médio do mar) deve coincidir com uma superfície equipotencial do campo de gravidade. A partir da igualdade de potenciais na superfície, combinada com o desenvolvimento em série da gravidade e do potencial centrífugo, obtém-se a relação formal de Clairaut. Ele também está intimamente ligado à Teoria do Elipsoide de Rotação e aparece em discussões sobre o geóide e o campo de gravidade normal.

%%%%%%%%%%%%%%%%%%%%%%%%%%%%%%%%%%%%%%%%%%%%%%%%%%%%%%%%%%%%%%%%%%%%%%%%%%%%%%%%%%%%%%%%%%%%%%%%%%%%%%%%%%%%%%%%%%%%%%%%%%%%%%%%%%%%%%%%%%%%%%%%%%%%%%%%%%%%%%%%%%%%%%%%%%%%%%%%%%%%%%%%%%%%%%%%%%%%%%%%%%%%%%%%%%%%%%%%%%%%%%%%%%%

\section{Utilizando os parâmetros do elipsoide GRS80, determine o valor de gravidade normal para um ponto localizado na latitude \(0^\circ 0^\prime 5^{\prime\prime}\) e altitude 2850 metros.}

A formula:


\[
\gamma_h = 
\frac{1}{W}
\left[
\underbrace{
\frac{Gm}{b'^2 + E^2}
}_{T_1}
+
\underbrace{
\frac{\omega^2 a^2 E q'}{(b'^2 + E^2) q_0}
}_{T_2}
\underbrace{
\left( \frac{\sin^2 \beta'}{2} - \frac{1}{6} \right)
}_{T_3}
-
\underbrace{
\omega^2 b' \cos^2 \beta'
}_{T_4}
\right]
\text{,}
\]



\noindent
como apresentada por Li \& Gotze (2001), fornece uma forma analítica para o cálculo da gravidade normal \(\gamma_h\) em qualquer ponto acima ou abaixo da superfície de um elipsoide de revolução, considerando os efeitos do achatamento e da rotação da Terra.



Para determinar o valor de \(\gamma_h\) em um ponto localizado na latitude geodésica \(\phi = 0^\circ 0' 5''\) e altitude elipsoidal \(h = 2850\;\text{m}\), foram utilizados os parâmetros do elipsoide de referência GRS80, listados na Tabela~\ref{tab:grs80}.

\begin{table}[h]
\centering
\caption{Parâmetros fundamentais do elipsoide GRS80.}
\label{tab:grs80}
\begin{tabular}{ll}
\toprule
Grandeza & Valor \\
\midrule
Semieixo maior (\(a\)) & \(6\,378\,137.0\;\text{m}\) \\
Semieixo menor (\(b\)) & \(6\,356\,752.3141\;\text{m}\) \\
Gravidade geocêntrica (\(Gm\)) & \(3.986005 \times 10^{14}\;\text{m}^3/\text{s}^2\) \\
Velocidade angular da Terra (\(\omega\)) & \(7.292115 \times 10^{-5}\;\text{rad/s}\) \\
Excentricidade linear (\(E = \sqrt{a^2 - b^2}\)) & \(521\,854.0097\;\text{m}\) \\
\bottomrule
\end{tabular}
\end{table}

\noindent
As coordenadas geodésicas dadas são:

\[
\phi = 0^\circ\,0'\,5'' = \dfrac{5}{3600} = 0{,}00138889^\circ \text{,} \quad h = 2850 \; \text{m} \text{.}
\]

\noindent
Primeiramente obtemos a latitude reduzida \(\beta\) no elipsoide de referência:

\[
\begin{aligned}
   \beta &= \tan^{-1}\left( \dfrac{b \cdot \sin \phi}{a \cdot \cos \phi} \right) \\
         &= \tan^{-1}\left[ \dfrac{6356752{,}314140356 \cdot \sin\left(\tfrac{5}{3600}\right)}{6378137 \cdot \cos\left(\tfrac{5}{3600}\right)} \right] \\
         &\approx 0{,}001384232207^\circ \text{.}
\end{aligned}
\]

\noindent
Calculam-se então os quadrados das distâncias entre o ponto e o plano equatorial (\(z_p^2\)), e entre o ponto e o eixo de rotação da Terra (\(r_p^2\)):



\[
\begin{aligned}
   z_p^2 &= \left( b \cdot \sin\beta + h \cdot \sin\phi \right)^2 \\
         &= \left[ 6356752{,}314140356 \cdot \sin(0{,}001384232207) + 2850 \cdot \sin\left(\tfrac{5}{3600}\right) \right]^2 \\
         &= 23606{,}62284572807 \; \text{m}^2 \text{,} \\
   r_p^2 &= \left( a \cdot \cos\beta + h \cdot \cos\phi \right)^2 \\
         &= \left[6378137 \cdot \cos(0{,}001384232207) + 2850 \cdot \cos\left(\tfrac{5}{3600}\right)\right]^2 \\
         &= 40716995070403{,}35 \; \text{m}^2 \text{.}
\end{aligned}
\]

\noindent
A partir das distâncias calculadas, obtêm-se os parâmetros auxiliares normalizados \(D\) e \(R\), definidos como:


\[
\begin{aligned}
   D &= \dfrac{r_p^2 - z_p^2}{E^2} \\
     &= \dfrac{40716995070403{,}35 - 23606{,}62284572807}{521854{,}009700248^2} \\
     &\approx 149{,}51255724414438 \text{,} \\
   R &= \dfrac{r_p^2 + z_p^2}{E^2} \\
     &= \dfrac{40716995070403{,}35 + 23606{,}62284572807}{521854{,}009700248^2} \\
     &\approx 149{,}51255741751115 \text{.}
\end{aligned}
\]

\noindent
Esses parâmetros são usados na definição do elipsoide auxiliar que passa exatamente pelo ponto de interesse. Utilizando os valores de \(D\) e \(R\), calcula-se o quadrado do cosseno da latitude reduzida \(\beta'\) no elipsoide auxiliar, por meio da seguinte expressão:

\[
\begin{aligned}
\cos^2 \beta' &= \dfrac{1}{2} + \dfrac{R}{2} - \sqrt{ \dfrac{1}{4} + \dfrac{R^2}{4} - \dfrac{D}{2} } \\
              &= \dfrac{1}{2} + \dfrac{149{,}51255741751115}{2} - \sqrt{ \dfrac{1}{4} + \dfrac{149{,}51255741751115^2}{4} - \dfrac{149{,}51255724414438}{2} } \\
              &\approx 0{,}9999999994163176 \text{.}
\end{aligned}
\]

\noindent
Então, a latitude reduzida \(\beta'\) no elipsoide auxiliar é obtida a partir de:

\[
\begin{aligned}
   \beta' &= \cos^{-1}\left( \sqrt{\cos^2 \beta'} \right) \\
          &= \cos^{-1}\left( \sqrt{0{,}9999999994163176} \right) \\
          &\approx 0{,}0013842385655864757^\circ \text{,}
\end{aligned}
\]

\noindent
e coordenada radial modificada \(b'\), correspondente ao semi-eixo menor do elipsoide auxiliar, é obtida por:



\[
\begin{aligned}
   b' &= \sqrt{r_p^2 + z_p^2 - E^2 \cdot \cos^2 \beta'} \\
      &= \sqrt{40716995070403{,}35 + 23606{,}62284572807 - 521854{,}009700248^2 \cdot 0{,}9999999994163176} \\
     &\approx 6359611{,}897492543 \; \text{m} \text{.}
\end{aligned}
\]

\noindent
Calcula-se os termos auxiliares \(q_0\) e \(q'\), que correspondem a funções que descrevem, respectivamente, a distribuição do potencial gravitacional no elipsoide de referência e no elipsoide auxiliar ajustado ao ponto de interesse.

%\left(\tfrac{5}{3600}\right)



\[
\begin{aligned}
   q_0
   &= \dfrac{1}{2} \left[\left(1 + \dfrac{3 \cdot b^2}{E^2}\right) \cdot \tan^{-1}\!\left(\dfrac{E}{b}\right) - \dfrac{3 \cdot b}{E}\right]\\
   &= \dfrac{1}{2} \left[\left(1 + 3 \cdot \dfrac{6356752{,}314140356^2}{521854{,}009700248^2}\right) \cdot \tan^{-1}\!\left(\dfrac{521854{,}009700248}{6356752{,}314140356}\right) - 3 \cdot \dfrac{6356752{,}314140356}{521854{,}009700248}\right]\\
   &\approx 0{,}00007334625840726972 \text{,}
\end{aligned}
\]


\[
\begin{aligned}
   q'&= 3\left(1 + \dfrac{b'^2}{E^2}\right)\left[1 -\dfrac{b'}{E} \cdot \tan^{-1}\left(\dfrac{E}{b'}\right)\right] -1\\
     &= 3\left(1 + \dfrac{6359611{,}897492543^2}{521854{,}009700248^2}\right)\left[1 -\dfrac{6359611{,}897492543}{521854.009700248} \cdot \tan^{-1}\left(\dfrac{521854.009700248}{6359611{,}897492543}\right)\right] -1\\
     &\approx 0{,}002685631457639115 \text{.}
\end{aligned}
\]

\noindent
Agora, para facilitar a equação pode ser escrita como:

\[
\gamma_h = \frac{1}{W} \left( T_1 + T_2 \cdot T_3 - T_4 \right)
\]

\noindent
em que os termos \(T_1\), \(T_2\), \(T_3\) e \(T_4\) são calculados como segue.

\[
\begin{aligned}
   T_1 &= \dfrac{Gm}{b'^2 + E^2} \\
       &= \dfrac{398600500000000}{6359611{,}897492543^2 + 521854{,}009700248^2} \\
       &\approx 9{,}789536263128698 \text{,}  \\[10pt]
   T_2 &= \dfrac{\omega^2 \cdot a^2 \cdot E \cdot q'}{(b'^2 + E^2)q_0} \\
       &= \dfrac{(7{,}292115 \times 10^{-5})^2 \cdot 6378137^2 \cdot 521854{,}009700248 \cdot 0{,}002685631457639115}{(6359611{,}897492543^2 + 521854{,}009700248^2)7.334625840726972 \times 10^{-5}} \\
       &\approx 0{,}10151645913370078 \text{,} \\[10pt]
   T_3 &= \dfrac{\sin^2\beta'}{2} - \dfrac{1}{6}\\
       &= \dfrac{5{,}836824357174919 \times 10^{-10}}{2} - \dfrac{1}{6}\\
       &\approx -0{,}16666666637482544 \text{,} \\[10pt]
   T_4 &= \omega^2 \cdot b' \cdot \cos^2\beta' \\
       &= (7{,}292115 \times 10^{-5})^2 \cdot 6359611{,}897492543 \cdot 0{,}9999999994163176 \\
       &\approx 0.033817198833632274 \text{,}
\end{aligned}
\]




\noindent
O fator de normalização \(W\) pode ser calculado da seguinte forma:

\[
\begin{aligned}
   W &= \sqrt{\dfrac{b'^2 + E^2 \cdot \sin^2\beta'}{b'^2 + E^2}} \\
     &= \sqrt{\dfrac{6359611{,}897492543^2 + 521854{,}009700248^2 \cdot 5.836824357174919 \times 10^{-10}}{6359611{,}897492543^2 + 521854{,}009700248^2}} \\
     &\approx 0{,}99665018867654 \text{.}
\end{aligned}
\]


\noindent
Somando os termos e aplicando o fator \(\frac{1}{W}\), temos:

\[
\begin{aligned}
   \gamma &= \dfrac{1}{W} \left( T_1 + T_2 \cdot T_3 - T_4 \right) \\
          &= \dfrac{ 9.789536263128698 + 0.10151645913370078 \cdot (-0.16666666637482544) - 0.033817198833632274}{0{,}99665018867654} \\
          &= 9{,}7715324\;\mathrm{m/s^2} \\
          &= 9{,}77153{,}24\;\mathrm{mGal} \text{.}
\end{aligned}
\]



\noindent
Este resultado coincide com o obtido por meio da biblioteca \texttt{boule}, que implementa a fórmula de Li \& Gotze (2001), validando a consistência do cálculo manual com a referência computacional.

\begin{lstlisting}[language=Python, caption={Cálculo da gravidade normal com a biblioteca \texttt{boule} 
para o modelo GRS80}]

Obtive 977153.24 no python

import boule as bl

h = 2850                       # altura elipsoidal em metros
lat = 5 / 3600                # latitude geodésica em graus decimais

gamma = bl.GRS80.normal_gravity(latitude=lat, height=h)
print(f"{gamma:.2f} mGal")

977153.24 mGal

Eu preciso demonstrar como se obtêm o resultado por uma resolução manual 
\end{lstlisting}

%%%%%%%%%%%%%%%%%%%%%%%%%%%%%%%%%%%%%%%%%%%%%%%%%%%%%%%%%%%%%%%%%%%%%%%%%%%%%%%%%%%%%%%%%%%%%%%%%%%%%%%%%%%%%%%%%%%%%%%%%%%%%%%%%%%%%%%%%%%%%%%%%%%%%%%%%%%%%%%%%%%%%%%%%%%%%%%%%%%%%%%%%%%%%%%%%%%%%%%%%%%%%%%%%%%%%%%%%%%%%%%%%%%

\section{Qual é a principal diferença entre “anomalia da gravidade” e “distúrbio da gravidade”? Explicar detalhadamente.}

A anomalia da gravidade, geralmente representada por \( \Delta g \), é definida como a diferença entre a gravidade observada \( g_P \), medida sobre a superfície física da Terra (em um ponto \( P \)), e a gravidade normal \( \gamma_Q \), calculada no ponto \( Q \) da superfície do elipsoide de referência que possui a mesma latitude geodésica e longitude que \( P \). Matematicamente, temos:

\[
\Delta g = g_P - \gamma_Q
\]

\noindent
Essa definição implica que a anomalia da gravidade compara valores obtidos em pontos diferentes: o valor observado na superfície física com o valor teórico que o modelo de elipsoide forneceria se o ponto estivesse sobre sua superfície. Como os pontos \( P \) e \( Q \) não coincidem espacialmente, a anomalia da gravidade reflete não apenas variações internas da estrutura da Terra (diferenças de densidade, por exemplo), mas também os efeitos do relevo e da topografia. 

Por outro lado, o distúrbio da gravidade, denotado por \( \delta g \), é definido como a diferença entre a gravidade observada \( g_P \), no ponto \( P \) sobre a superfície física da Terra, e a gravidade normal \( \gamma_P \) calculada no mesmo ponto físico \( P \):

\[
\delta g = g_P - \gamma_P
\]

\noindent
Aqui, a comparação é feita no mesmo ponto do espaço, o que significa que o distúrbio da gravidade elimina o efeito do desnível entre a superfície física e o elipsoide. Ele representa, portanto, uma medida local e intrínseca da diferença entre o campo real e o campo de referência.

%%%%%%%%%%%%%%%%%%%%%%%%%%%%%%%%%%%%%%%%%%%%%%%%%%%%%%%%%%%%%%%%%%%%%%%%%%%%%%%%%%%%%%%%%%%%%%%%%%%%%%%%%%%%%%%%%%%%%%%%%%%%%%%%%%%%%%%%%%%%%%%%%%%%%%%%%%%%%%%%%%%%%%%%%%%%%%%%%%%%%%%%%%%%%%%%%%%%%%%%%%%%%%%%%%%%%%%%%%%%%%%%%%%

\section{O que estabelece a “Equação fundamental da Geodésia Física”? Explicar detalhadamente.}

A Equação Fundamental da Geodésia Física estabelece uma relação direta entre a anômala da gravidade e o potencial anômalo, permitindo interpretar variações observadas no campo da gravidade terrestre como decorrentes de diferenças entre o campo real de gravidade e o campo idealizado de referência. 

Formalmente, a equação fundamental da geodesia física pode ser expressa como:

\[
\Delta g = -\frac{\partial T}{\partial h} + \frac{1}{\gamma} \frac{\partial U}{\partial h} \left( \frac{\partial T}{\partial h} \right) - \frac{\partial \gamma}{\partial h} \, \zeta
\]

\noindent
ou, em uma forma aproximada de primeira ordem, comumente usada na prática:

\[
\Delta g \approx -\frac{\partial T}{\partial h} - \frac{\partial \gamma}{\partial h} \, \zeta
\]

\noindent
onde \( \Delta g \) é a anomalia da gravidade, definida como a diferença entre a gravidade observada e a gravidade normal no elipsoide; \( T = W - U \) é o potencial anômalo, correspondente à diferença entre o geopotencial real \( W \) e o potencial normal \( U \); \( h \) é a altura elipsoidal, medida ao longo da normal ao elipsoide; \( \gamma \) é a gravidade normal no elipsoide; \( \frac{\partial T}{\partial h} \) representa a derivada vertical do potencial anômalo; \( \frac{\partial \gamma}{\partial h} \) é o gradiente vertical da gravidade normal; e \( \zeta \) é a ondulação do geóide, ou seja, a distância entre o geóide e o elipsoide de referência.

A equação mostra que a anomalia da gravidade pode ser interpretada como composta por duas partes principais: uma relacionada à variação do potencial anômalo com a altura (isto é, o campo de massa acima e abaixo do ponto considerado), e outra relacionada à diferença geométrica entre a superfície de nível físico (o geóide) e o elipsoide de referência. Em termos físicos, a equação expressa como uma diferença no campo de gravidade pode ser decomposta em um efeito devido a massas anômalas (representadas por \( \partial T/\partial h \)) e a uma perturbação geométrica na posição das superfícies equipotenciais (representada por \( \zeta \)).

A derivação dessa equação parte da comparação entre o campo de gravidade real \( g \), associado ao geopotencial \( W \), e o campo de gravidade normal \( \gamma \), associado ao potencial normal \( U \). Sabendo que:

\[
g = -\nabla W \quad \text{e} \quad \gamma = -\nabla U \text{,}
\]

\noindent
e que o potencial anômalo é a diferença \( T = W - U \), então o vetor diferença entre os dois campos de gravidade será dado por:

\[
\vec{g} - \vec{\gamma} = -\nabla T \text{.}
\]

\noindent
Considerando a componente vertical dessa diferença, tem-se que a variação da gravidade medida em relação à gravidade normal (isto é, \( \Delta g \)) pode ser interpretada como o efeito da derivada vertical de \( T \), mais um termo corretivo associado ao deslocamento entre superfícies de referência. Na prática, essa equação permite converter observações gravimétricas em estimativas do potencial anômalo, que por sua vez é utilizado para o cálculo da altura geoidal, para a modelagem do geóide e para o ajuste de redes altimétricas.